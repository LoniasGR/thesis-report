% Title page
\maketitle

% Intentionally blank page
\blankpagecount

% Examination Page
\examinationpage

% Copyrights page
\copyrightspage

% After here everything is center-aligned
\justifying


%%%%%%%%%%%%%%%%%%%%%%%%%%%%%%%%%%%%%%%%%%%%%%%%%%%%%%%%%%%%%%%%%
%% Περίληψη
%%%%%%%%%%%%%%%%%%%%%%%%%%%%%%%%%%%%%%%%%%%%%%%%%%%%%%%%%%%%%%%%%
\chapter*{Περίληψη}
\addcontentsline{toc}{chapter}{Περίληψη}
Η δημιουργία συστάσεων είναι ένα πρόβλημα στο οποίο έχει δείξει
ενδιαφέρον τόσο η ακαδημαική κοινότητα, όσο και η βιομηχανία. Πολλές επιτυχημένες εφαρμογές έχουν χτιστεί πάνω στην δημιουργία καλών
συστάσεων, όπως ο αλγόριθμος του \en{Netflix}, αλλά και οι συστάσεις προϊόντων της \en{Amazon}.
Η άνθιση προέρχερται κυρίως λόγο του μεγάλου όγκου πληροφοριών που υπάρχουν στο
διαδίκτυο, του οποίου η αναζήτηση και επιμέλεια χειροκίνητα από ανθρώπους είναι αδύνατη.

\noindent
Επιπλέον, η ενισχυτική
μάθηση είναι ένας απο τους πλέον διαδεδομένους τρόπους εκπαίδευσης
πρακτόρων, κυρίως σε περιβάλλοντα που το τελικό αποτέλεσμα γίνεται
γνωστό μετά από πολλά βήματα, και δεν υπάρχει γνωστή βέλτιστη λύση για κάθε βήμα.

\noindent
Τέλος η χρήση μηχανική μάθησης για την παραγωγή και την κατανόηση
κειμένου είναι ένας κλάδος ο οποίος έχει δει μεγάλη άνθιση τα
τελευταία λίγα χρόνια

\noindent
Η τρέχουσα διπλωματική ασχολείται με την δημιουργία ενός συστήματος συστάσεων, το οποίο δουλεύει παράλληλα με ένα διαλογικό σύστημα, το
οποίο προτείνει θέματα συζήτησης στον χρήστη. Η επιλογή των πρακτόρων
έγινε με βάση την γνώση ότι το περιβάλλον εργασίας περιείχε περιορισμένα δεδομένα, καθώς και με βάση το γεγονός ότι
ένας από τους στόχους ήταν η προσαρμογή των συστάσεων ανάλογα με την χρονική περίοδο και τις αλλαγές στις ανάγκες που προκύπτουν με βάση αυτή,
οπότε κρίθηκε η χρήση τεχνικών ενισχυτικής μάθησης ως η βέλτιστη λύση.
\vspace{20ex}
\section*{Λέξεις Κλειδιά}
\en{TODO}


\clearpage

\blankpagecount

%%%%%%%%%%%%%%%%%%%%%%%%%%%%%%%%%%%%%%%%%%%%%%%%%%%%%%%%%%%%%%%%%
%% Abstract
%%%%%%%%%%%%%%%%%%%%%%%%%%%%%%%%%%%%%%%%%%%%%%%%%%%%%%%%%%%%%%%%%
\chapter*{\en{Abstract}}
\addcontentsline{toc}{chapter}{\en{Abstract}}

\begin{otherlanguage}{english}
    \en{TODO}

    \vspace{20ex}
    \section*{\en{Keywords}}
    \en{TODO}
\end{otherlanguage}
\clearpage

\blankpagecount

%%%%%%%%%%%%%%%%%%%%%%%%%%%%%%%%%%%%%%%%%%%%%%%%%%%%%%%%%%%%%%%%%
%% Ευχαριστίες
%%%%%%%%%%%%%%%%%%%%%%%%%%%%%%%%%%%%%%%%%%%%%%%%%%%%%%%%%%%%%%%%%
\chapter*{Ευχαριστίες}
\addcontentsline{toc}{chapter}{Ευχαριστίες}

Ευχαριστώ την οικογένεια μου και τους καθηγητές μου.
\clearpage

\blankpagecount


%%%%%%%%%%%%%%%%%%%%%%%%%%%%%%%%%%%%%%%%%%%%%%%%%%%%%%%%%%%%%%%%%
%% TABLE OF CONTENTS (TOC), LISTS OF FIGURES, TABLES, ETC.
%%%%%%%%%%%%%%%%%%%%%%%%%%%%%%%%%%%%%%%%%%%%%%%%%%%%%%%%%%%%%%%%%

\tableofcontents

\listoffigures

\listoftables

\clearpage

