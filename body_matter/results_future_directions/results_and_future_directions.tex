\chapter{Αποτελέσματα και περαιτέρω εργασία}
\label{chap:results}
\section{Τελευταίες Τεχνολογίες (\en{State of the Art})}

Οι τελευταίες τεχνολογίες οσον αφορά τον διάλογο εμφανίστηκαν αφού είχαμε ξεκινήσει την εργασία μας. Αυτή είναι το \en{ChatGPT}, το οποίο δημιουργήθηκε από την \en{OpenAI}, το οποίο έφερε μια επανάσταση στο πως αντιλαμβανόμαστε τους διαλογικούς πράκτορες και το τι μπορούν να κάνουν. Συγκεκριμένα το \en{ChatGPT} μπορεί να παρέχει αναλυτικές και καλά δομημένες απαντήσεις σε πολλές κατηγορίες θεμάτων, μπορεί να κρατήσει υπόψιν του τι έχει συζητηθεί νωρίτερα στην συζήτηση και παρέχει απαντήσεις σε πολλές γλώσσες. Παρ'όλα αυτά μπορεί να παρέχει με βεβαιότητα απαντήσεις οι οποίες είναι ανακριβείς ή και τελείως λάθος. Το \en{ChatGPT} βασίζεται στα γλωσσικά μοντέλα \en{GPT3.5} και \en{GPT4}, τα οποία είναι κάποια από τα μεγαλύτερα που υπάρχουν αυτή την στιγμή. Το \en{ChatGPT} οδήγησε στην διάδοση της προόδου της Τεχνητής Νοημοσύνης και του τι μπορεί να καταφέρει και έφερε στο προσκήνιο πολλές συζητήσεις σχετικά με την ταχύτητα εξέλιξης της και τις κοινωνικές επιπτώσεις που μπορεί να έχει η τεχνολογία.
Μετά την εμφάνιση του \en{ChatGPT}, εμφανίστηκαν επίσης και άλλα παρόμοια μοντέλα όπως το \en{Bard} από την \en{Google}, το οποίο βασίζεται στο γλωσσικό μοντέλο \en{PaLM}, το οποίο έχει 540 δισεκατομμύρια παραμέτρους, καθώς και το γλωσσικό μοντέλο \en{LLaMA} από την \en{Meta}, το οποίο έχει στην μεγαλύτερη του έκδοση 65 δισεκατομμύρια παραμέτρους.
Οσον αφορά τις τελευταίες τεχνολογίες στο κομμάτι των συστάσεων, ένα μεγάλο ρεύμα έρευνας έχει επικεντρωθεί στην χρήση τεχνικών \en{Multi-Armed Bandits} για την παραγωγή των συστάσεων. Οι τεχνικές αυτές ουσιαστικά είναι ύπο-κλάδος της Ενισχυτικής Μάθησης, ο οποίος προσπαθεί να λύσει ένα πιο απλοποιημένο πρόβλημα. Στη βιομηχανία, μεγάλες εταιρίες τεχνολογίας χρησιμοποιούν τεχνικές \en{bandits} για να λύσουν προβλήματα όπως η σύσταση νέων να δείξουν στην αρχική σελίδα τους, ή των εξωφυλλων στις ταινίες που είναι προτεινόμενες για τον χρήστη. Αυτή είναι η προσέγγιση που ακολουθήσαμε και εμείς για να λύσουμε το πρόβλημα της σύστασης. Συγκεκριμένα δοκιμάσαμε διάφορους αλγορίθμους και προσπαθήσαμε να εντοπίσουμε ποιοί ανταπεξέρχονται καλύτερα, με βάση το γεγονός ότι ο όγκος προηγούμενων δεδομένων που έχουμε είναι περιορισμένος.


\section{Αποτελέσματα}
\section{Επεκτάσεις}