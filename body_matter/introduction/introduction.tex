\chapter{Εισαγωγή}

Η εργασία αυτή ασχολείται με την ανάπτυξη ενός συστήματος το οποίο συνδέει τρία βασικά στοιχεία. Τις συστάσεις, τα διαλογικά συστήματα και την ενισχυτική μάθηση. Κάθε ένα από αυτά τα στοιχεία θα αναλυθεί σε βάθος στα επόμενα κεφάλαια. Σε αυτή την ενότητα θα γίνει μια εισαγωγή στα κίνητρα, τις βασικές ιδέες που χρησιμοποιήθηκαν στην εργασία, καθώς και τα κύρια χαρακτηριστικά των συστημάτων αυτών και των προϋπαρχόντων υποδομών.


\section{Κίνητρα}

Οι διαλογικοί πράκτορες δημιουργήθηκαν με σκοπό να μειώσουν τον ανθρώπινο κόπο. Αντί να χρειάζεται να υπάρχει ένας άνθρωπος-πράκτορας πάντα διαθέσιμος να εξυπηρετήσει τα ερωτήματα που έρχονται από πελάτες/ενδιαφερόμενους, το όραμα ήταν να υπάρχουν αυτόματα διαλογικά συστήματα που τους εξυπηρετούν με παρόμοια ποιότητα υπηρεσίας. Η έρευνα στα διαλογικά συστήματα ξεκίνησε από τις αρχές του 1970 με το σύστημα βασισμένο σε κανόνες που ονομαζόταν \en{ELISA} και έχει φτάσει στο απόγειο της πλεον, με την δημιουργία του συστήματος \en{ChatGPT}, ενός διαλογικού συστήματος γενικού σκοπού βασισμένο στην αρχιτεκτονική των μετασχηματιστών (\en{transformers}) και έχει 175 δισεκατομμύρια παραμέτρους. Παρόλα αυτά, η διατήρηση του ενδιαφέροντος του χρήστη κατά την διάρκεια του διαλόγου και η συνέχιση του, είναι ένα πρόβλημα που δεν έχει λυθεί επαρκώς.

Σκοπός της διπλωματικής εργασίας είναι η σύσταση ενδιαφερόντων θεμάτων στον χρήστη του διαλογικού συστήματος Θεανώ.Τα ενδιαφέροντα θέματα ορίζονται ως τα θέματα τα οποία θα επιτύγχαναν να διατηρήσουν την αλληλεπίδραση με τον χρήστη για μεγαλύτερο αριθμό γύρων. Έτσι για παράδειγμα, ένας χρήστης θα ρωτήσει κάποια ερώτηση, και αφού το σύστημα απαντήσει, θα του προτείνει και ένα θέμα περαιτέρω συζήτησης, με βάση τα θέματα τα οποία μπορεί να απαντήσει. Πιο συγκεκριμένα, η διπλωματική εργασία χτίζεται πάνω στο σύστημα Θεανώ, του ΕΚ Αθηνά, το οποίο είναι ένα διαλογικό σύστημα με θέμα τον \en{Covid-19}. Περαιτέρω επεξήγηση της Θεανώς, θα γίνει στις επόμενες ενότητες.

Οπως είναι προφανές, κάθε χρήστης έχει διαφορετικό ιστορικό, διαφορετικά ενδιαφέροντα και διαφορετικές πληροφορίες που την ενδιαφέρουν. Πολλές από αυτές τις πληροφορίες, δεν είναι ποτέ διαθέσιμες στο σύστημα μας, ενώ άλλες είναι διαθέσιμες αφού η Θεανώ αρχίσει να επικοινωνεί με τον χρήστη. Έτσι το σύστημα θα πρέπει να μπορεί να κάνει δύο πράγματα. Το πρώτο είναι η δημιουργία μιας γενικής κατανόησης των θεμάτων που σχετίζονται νοηματικά και είναι πιθανό ότι αν ο χρήστης ρωτήσει για το ένα, να ενδιαφέρεται και για το άλλο. Το δεύτερο είναι η κατανόηση των ενδιαφερόντων του χρήστη με βάση τις ερωτήσεις που κάνει και τα θέματα που συμφωνεί να μάθει ή όχι. Καθώς οι χρήστες δεν είναι γνωστοί από πριν, και τα ενδιαφέροντα θέματα μπορεί να αλλάζουν ανα περίοδο, η χρήση κλασικής επιτηρούμενης μάθησης δεν είναι εφικτή. Αυτό συμβαίνει γιατί στην επιτηρούμενη μάθηση χρειάζονται γνωστά δείγματα και σωστές απαντήσεις σε αυτά δείγματα, οι οποίες δεν υπάρχουν σε αυτή την περίπτωση, καθώς δεν υπάρχει μια ξεκάθαρη ανάγκη για τους χρήστες. Αντίθετα χρειάζεται μια πιο δυναμική προσαρμογή, η οποία επιτυγχάνεται μέσω την σύγχρονης εκμάθησης \en{(online learning)} και της ενισχυτικής μάθησης \en{(reinforcement learning)}.

\section{Η Θεανώ}

Η Θεανώ είναι ένας διαλογικός πράκτορας που έχει σκοπό την ενημέρωση σχετικά με τον \en{Covid-19}. Βασίζεται πάνω στα εργαλεία που προσφέρονται από την εργαλειοθήκη του \en{Rasa}, η οποία παρέχει ένα ολοκληρωμένο διαλογικό σύστημα από άκρη σε άκρη. Συγκεκριμένα, παρέχει σύστημα κατανόησης της φυσικής γλώσσας, αναγνώρισης των προθέσεων του συνομιλητή και επιλογής της κατάλληλης απάντησης με βάση αυτό. Τέλος, έχει και ένα συστημα παραγωγής φυσικής γλώσσας για την απάντηση. Επιπλέον, κάθε πρόθεση μπορεί να αντιστοιχιστεί σε μια πράξη και με βάση αυτή να δημιουργηθεί η απάντηση του πράκτορα. Έτσι το σύστημα είναι αρκετά εύρωστο ώστε να μπορεί να ανταποκριθεί τόσο σε αιτήματα του χρήστη που είναι πιο γενικού σκοπού, αλλά και συγκεκριμένα αιτήματα. H Θεανώ εκπαιδεύεται τόσο από πλασματικά δεδομένα, ειδικά στην αρχή, όσο και από δεδομένα με βάση της συνομιλίες που έχει κάνει με τους χρήστες. Η εκπαίδευση ουσιαστικά γίνεται με την μέθοδο της επιβλεπόμενης μάθησης, και συγκεκριμένα είναι ένα είδος ταξινόμησης.

Έτσι η Θεανώ μπορεί να απαντήσει ερωτήσεις σχετικά με τα εμβόλια, την κατάσταση των εμβολιασμών τόσο στην Ελλάδα όσο και στο εξωτερικό, την κατάσταση των ΜΕΘ, και άλλα παρόμοια θέματα. Επιπλέον, η Θεανώ, πριν την δική μας συνεισφορά, είχε μια δράση, η οποία ήταν υπεύθυνη κάποιες φορές - με βάση την τύχη, να προτείνει ένα τυχαίο θέμα για περαιτέρω συζήτηση μετά την απο την απάντηση κάποιας πρόθεσης του χρήστη. Έτσι προσπαθούσε να "κρατήσει" τον χρήστη περισσότερο στην συζήτηση και να τον βοηθήσει να μάθει περισσότερα πράγματα.

Στόχος μας στην εργασία ήταν η βελτίωση του συστήματος αυτών των συστάσεων με στόχο την διατήρηση του ενδιαφέροντος του χρήστη για περισσότερους γύρους.

\section{Τελευταίες Τεχνολογίες (\en{State of the Art})}

Οι τελευταίες τεχνολογίες οσον αφορά τον διάλογο εμφανίστηκαν αφού είχαμε ξεκινήσει την εργασία μας. Αυτό είναι το \en{ChatGPT}, το οποίο έφερε μια επανάσταση στο πως αντιλαμβανόμαστε τους διαλογικούς πράκτορες και το τι μπορούν να κάνουν.
