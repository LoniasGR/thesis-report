\chapter{Εισαγωγή}

Η εργασία αυτή ασχολείται με την ανάπτυξη ενός συστήματος το οποίο συνδέει τρία βασικά στοιχεία:
τις συστάσεις, τα διαλογικά συστήματα και την ενισχυτική μάθηση.
Κάθε ένα από αυτά τα στοιχεία θα αναλυθεί σε βάθος στα επόμενα κεφάλαια.
Σε αυτή την ενότητα θα γίνει μια εισαγωγή στα κίνητρα, τις βασικές ιδέες που χρησιμοποιήθηκαν στην εργασία,
καθώς και τα κύρια χαρακτηριστικά των συστημάτων αυτών και των προϋπάρχουσων υποδομών.


\section{Κίνητρα}

Οι διαλογικοί πράκτορες δημιουργήθηκαν με σκοπό να μειώσουν τον ανθρώπινο κόπο.
Αντί να χρειάζεται να υπάρχει ένας άνθρωπος-πράκτορας πάντα διαθέσιμος να εξυπηρετήσει τα ερωτήματα που έρχονται από πελάτες/ενδιαφερόμενους,
το όραμα ήταν να υπάρχουν αυτόματα διαλογικά συστήματα που τους εξυπηρετούν με παρόμοια ποιότητα υπηρεσίας. Η έρευνα στα διαλογικά συστήματα
ξεκίνησε από τις αρχές του 1970 με το σύστημα βασισμένο σε κανόνες που ονομαζόταν \en{ELIZA}\cite{eliza} και έχει φτάσει στο απόγειο της πλεον, με την
δημιουργία του συστήματος \en{ChatGPT}, ενός διαλογικού συστήματος γενικού σκοπού βασισμένο στην αρχιτεκτονική των μετασχηματιστών (\en{transformers})
και έχει 175 δισεκατομμύρια παραμέτρους\cite{brown2020language}. Παρόλα αυτά, η διατήρηση του ενδιαφέροντος του χρήστη κατά την διάρκεια του διαλόγου και η συνέχιση του, είναι
ένα πρόβλημα που δεν έχει λυθεί επαρκώς.

Σκοπός της διπλωματικής εργασίας είναι η σύσταση ενδιαφερόντων θεμάτων στον χρήστη του διαλογικού συστήματος Θεανώ, το οποίο δημιουργήθηκε στο ΕΚ Αθηνά. Η
Θεανώ είναι ένα διαλογικό σύστημα, το οποίο μπορεί να απαντήσει καίρια ερωτήματα σχετικά με τον \en{Covid-19}. Περαιτέρω επεξήγηση της Θεανώς
θα γίνει στις επόμενες ενότητες. Τα ενδιαφέροντα θέματα ορίζονται ως τα θέματα τα οποία θα επιτύγχαναν να διατηρήσουν την αλληλεπίδραση με τον χρήστη για μεγαλύτερη διάρκεια.
Έτσι, για παράδειγμα, ένας χρήστης θα ρωτήσει κάποια ερώτηση, και αφού το σύστημα απαντήσει,
θα του προτείνει και ένα θέμα περαιτέρω συζήτησης, με βάση τα θέματα τα οποία μπορεί να απαντήσει.

Όπως είναι προφανές, κάθε χρήστης έχει διαφορετικό ιστορικό, διαφορετικά ενδιαφέροντα και διαφορετικές πληροφορίες που τον ενδιαφέρουν.
Πολλές από αυτές τις πληροφορίες δεν είναι ποτέ διαθέσιμες στο σύστημα μας, ενώ άλλες είναι διαθέσιμες αφού η Θεανώ αρχίσει να επικοινωνεί με τον χρήστη.
Έτσι, το σύστημα θα πρέπει να μπορεί να κάνει δύο πράγματα. Το πρώτο είναι να μπορεί να κατανοήσει τα θέματα που σχετίζονται νοηματικά και
είναι πιθανό ότι αν ο χρήστης ρωτήσει για το ένα, να ενδιαφέρεται και για το άλλο. Το δεύτερο είναι να μπορεί να κατανοήσει τα ενδιαφέροντα του χρήστη με βάση τις
ερωτήσεις που κάνει και τα θέματα που επιθυμεί να μάθει ή όχι. Καθώς οι χρήστες δεν είναι γνωστοί από πριν, και τα ενδιαφέροντα θέματα μπορεί να
αλλάζουν ανα περίοδο, η χρήση κλασικής επιβλεπόμενης μάθησης δεν είναι εφικτή. Αυτό συμβαίνει γιατί στην επιβλεπόμενη μάθηση χρειάζονται γνωστά
δείγματα και σωστές απαντήσεις σε αυτά δείγματα, οι οποίες δεν υπάρχουν σε αυτή την περίπτωση, καθώς δεν υπάρχει μια ξεκάθαρη ανάγκη για τους χρήστες.
Αντίθετα χρειάζεται μια πιο δυναμική προσαρμογή, η οποία επιτυγχάνεται μέσω την σύγχρονης μάθησης \en{(online learning)} και της ενισχυτικής μάθησης
\en{(reinforcement learning)}.

\section{Η Θεανώ}

Η Θεανώ είναι ένας διαλογικός πράκτορας που έχει σκοπό την ενημέρωση σχετικά με τον \en{Covid-19}.\cite{ventoura-etal-2021-theano}
Βασίζεται πάνω στα εργαλεία που προσφέρονται από την εργαλειοθήκη του \en{Rasa}, η οποία παρέχει ένα ολοκληρωμένο διαλογικό
σύστημα από άκρη σε άκρη. Συγκεκριμένα, παρέχει σύστημα κατανόησης της φυσικής γλώσσας, αναγνώρισης των προθέσεων του συνομιλητή και
επιλογής της κατάλληλης απάντησης με βάση αυτό. Τέλος, έχει και ένα συστημα παραγωγής φυσικής γλώσσας για την απάντηση. Επιπλέον, κάθε
πρόθεση μπορεί να αντιστοιχιστεί σε μια πράξη και με βάση αυτή να δημιουργηθεί η απάντηση του πράκτορα. Έτσι το σύστημα είναι αρκετά
εύρωστο ώστε να μπορεί να ανταποκριθεί τόσο σε αιτήματα του χρήστη που αποσκοπούν στην επίτευξη κάποιου συγκεκριμένου στόχου όσο και σε άλλα που είναι πιο γενικά.
H Θεανώ εκπαιδεύεται τόσο με την χρήση συνθετικών δεδομένων, ειδικά στην αρχή, όσο και με δεδομένα από συνομιλίες που έχει κάνει με τους χρήστες.
Η εκπαίδευση γίνεται με την μέθοδο της επιβλεπόμενης μάθησης, και συγκεκριμένα είναι ένα είδος ταξινόμησης. Περισσότερες πληροφορίες για την
εκπαίδευση της Θεανώς υπάρχουν στο Κεφάλαιο \ref{chap:dialogue_and_recommendations}.

Έτσι η Θεανώ μπορεί να απαντήσει ερωτήσεις σχετικά με τα εμβόλια, την κατάσταση των εμβολιασμών τόσο στην Ελλάδα όσο και στο εξωτερικό,
την κατάσταση των ΜΕΘ, και άλλα παρόμοια θέματα. Επιπλέον, η Θεανώ, στην βασική της έκδοση, είχε την δυνατότητα να προτείνει τυχαία θέματα για να
συνεχίσει την συζήτηση, αφού ο χρήστης ρωτήσει κάτι. Έτσι προσπαθούσε να ((κρατήσει)) τον χρήστη περισσότερο στην συζήτηση και να τον βοηθήσει
να μάθει περισσότερα πράγματα. Η χρήση αυτών των συστάσεων δείχνει να συνεισφέρει στην μεγαλύτερη διάρκεια των διαλόγων μεταξύ
χρήστη και Θεανώς. Όμως, η συχνή τους χρήση συνδέεται με χαμηλότερο αίσθημα ότι η Θεανώ καταλαβαίνει τον χρήστη.

Στόχος μας στην εργασία είναι η βελτίωση του συστήματος αυτών των συστάσεων με στόχο να επιτύχουμε τόσο την μεγαλύτερη διάρκεια των διαλόγων, αλλά ταυτόχρονα
να αυξήσουμε και το αίσθημα ότι η Θεανώ συναισθάνεται τον χρήστη.

\section{Η συνεισφορά μας}

Σκοπός της εργασίας είναι η δημιουργία ενός συστήματος συστάσεων το οποίο θα διαλειτουργεί με το προυπάρχον σύστημα της Θεανώς και θα παρέχει διαλογικές
συστάσεις στα θέματα που θα κρατήσουν το ενδιαφέρον του χρήστη για παραπάνω διαλογικούς γύρους.
Για την εφαρμογή των μεθόδων \en{bandits} στο πρόβλημα, χρησιμοποιήσαμε το εργαλείο \en{Vowpal Wabbit}, το οποίο προσφέρει έτοιμες πολιτικές \en{bandits},
καθώς και εργαλεία για σύγχρονη εκμάθηση. Επιπλέον, για την καλύτερη ροή πληροφορίας μεταξύ του συστήματος συστάσεων και της Θεανώς, τα δύο αυτά συστήματα
διαχωρίστηκαν, και δημιουργήθηκε μια νέα μίκρο-υπηρεσία (\en{micro-service}) η οποία είναι υπεύθυνη για τις συστάσεις. Αυτό τελικά σημαίνει ότι η υπηρεσία
και η λειτουργικότητα των συστάσεων είναι μεταφέρσιμη και σε άλλα περιβάλλοντα.
Το σύστημα συστάσεων εκπαιδεύτηκε αρχικά με την χρήση ασύγχρονης εκμάθησης (\en{offline learning}) και μετέπειτα εκπαιδεύεται μέσω της αλληλεπίδρασης του
με τους χρήστες της Θεανώς. Έτσι μπορεί να ακολουθήσει τα ρεύματα και τα ενδιαφέροντα των χρηστών καθώς αυτά αλλάζουν με τον χρόνο.
Τέλος, η υπηρεσία των συστάσεων σχεδιάστηκε με τρόπου που να είναι εύκολα μεταφέρσιμο σε άλλους διαλογικούς πράκτορες σχεδιασμένους με τα εργαλεία του
\en{Rasa}, με μικρές τροποποιήσεις στον περιβάλλοντα κώδικα.

\section{Διάρθρωση της εργασίας}

Στο Κεφάλαιο \ref{chap:rl} παρουσιάζονται οι βασικές γνώσεις Ενισχυτικής Μάθησης. Το κεφάλαιο αυτό παρέχει όλες τις απαραίτητες πληροφορίες για την κατανόηση της
λειτουργίας των τεχνικών \en{bandits}, οι οποίες παρουσιάζονται αναλυτικότερα στο Κεφάλαιο \ref{chap:bandits}, μαζί με τους βασικότερους αλγορίθμους που χρησιμοποιούνται σήμερα.
Έπειτα, στο Κεφάλαιο \ref{chap:dialogue_and_recommendations} παρουσιάζονται συνοπτικά οι βασικές ιδέες γύρω από τα διαλογικά συστήματα και την λειτουργία τους. Στο Κεφάλαιο \ref{chap:ourwork}
παρουσιάζεται αναλυτικά η δική μας εργασία και συνεισφορές. Κλείνοντας, στο Κεφάλαιο \ref{chap:results}, προτείνονται ιδέες για περαιτέρω εξερεύνηση, καθώς και τα σημαντικότερα αποτελέσματα.
